
\section{Introduction}
\label{sec:intro}

The rise of ``Web 2.0'' applications delivering dynamic, highly
interactive user experiences has been accompanied by a new generation
of programming frameworks~\cite{web20}. These frameworks simplify
common tasks such as content templating and presentation, request
handling, and, notably, data storage, allowing developers to focus on
``agile'' development of their applications. These frameworks embody
the most recent realization of the vision of object-relational mapping
(ORM) systems~\cite{orm-db}, albeit at a unprecedented scale of deployment and
programmer adoption.

A central player in modern frameworks is Ruby on Rails (or, simply
``Rails'')~\cite{rails-book,rails-computer}, an open source codebase
powering sites including (at one point) Twitter~\cite{twitter-rails},
Airbnb~\cite{airbnb-rails}, GitHub~\cite{github-rails},
Hulu~\cite{hulu-rails}, Shopify~\cite{shopify-rails},
Groupon~\cite{groupon-rails}, SoundCloud~\cite{soundcloud-rails},
Twitch~\cite{twitch-rails}, Goodreads~\cite{goodreads-rails}, and
Zendesk~\cite{zendesk-rails}. From the perspective of database systems
research, Rails is interesting for at least two reasons. First, it
continues to be a popular means of developing responsive web
application front-end and business logic, with an active open source
community and user base. Rails recently celebrated its tenth
anniversary and enjoys considerable commercial interest, both in terms
of deploy base and the availability of hosted ``cloud'' deployment
environments such as Heroku. Thus, Rails programmers represent a large
class of consumers of database technology. Second, and perhaps more
importantly, Rails is ``opinionated
software''~\cite{dhh-opinionated}. That is, Rails embodies the strong
personal convictions of its creator, David Heinemeier Hansson (DHH),
and developer community and is particularly opinionated towards the
database systems that Rails tasks with data storage. To quote DHH:
\begin{quote}
``I don't \textit{want} my database to be clever! \dots I consider stored procedures and constraints vile and reckless destroyers of coherence. No, Mr. Database, you can not have my business logic. Your procedural ambitions will bear no fruit and you'll have to pry that logic from my dead, cold object-oriented hands \dots I want a single layer of cleverness: My domain model.''~\cite{dhh-clever}
\end{quote}
Thus, this wildly successful software framework bears an actively
antagonistic relationship to database management systems, echoing a familiar refrain of the ``NoSQL'' movement: get the database out of the way and let the application do the work.

In this paper, we examine the implications of this decision---for
Rails as a framework, for applications built using Rails, and for
database systems builders---through the lens of data integrity and
concurrency control. In particular, by shunning decades of work on
native database concurrency control solutions, Rails has developed a
set of primitives for handling application integrity in the
application tier---building, from the underlying database system's
perspective, a \textit{feral} concurrency control system. We examine
the design and, more importantly, \textit{use} of these feral
mechanisms and evaluate their effective ``cleverness'' in
practice. Our goal is to understand how this growing class of
applications currently interacts (or, as the case may be, does not
interact) with database systems and how we, as a database systems
community, we can positively engage with these criticisms to better
serve the needs of these developers.

We begin by surveying the state of Rails' application-tier concurrency
control primitives and examining their use in 67 open source
applications representing a variety of use cases from e-Commerce to
Customer Relationship Management and social networking. We find that,
overwhelmingly---instead of using traditional mechanisms like
transactions---these applications use Rails' built-in support for
declarative invariants, or \textit{validations}, to protect data
integrity. We find over $9700$ uses of application-level constraints
across the applications used to guard against referential integrity,
uniqueness, and data formatting violations.

Given this corpus, we subsequently ask: are these validations actually
correct? Do they work in practice? Under concurrent execution, Rails
will execute validation checks in parallel, and any validation logic
that is susceptible to races may lead to data corruption. Accordingly,
we apply invariant confluence analysis~\cite{coord-avoid} and show
that, in fact, over $74.7\%$ of Rails validations usage by volume is
actually safe. However, the remainder, which include uniqueness
violations under insertion and foreign key constraint violations under
deletion, are not. Hence, we quantify the impact of concurrency on
data corruption for Rails uniqueness and foreign key constraints under
both worst-case analysis and via actual Rails deployment. We
demonstrate that, for pathological workloads, validations reduce the
severity of data corruption by orders of magnitude but, nevertheless,
still permit integrity violations.

Given these results, we present recommendations for the database
research community. We survey several additional web frameworks and
demonstrate that many also provide a notion of feral validations,
suggesting an industry-wide trend. While the success of Rails and its
ilk are firm evidence of the continued ``impedance mismatch'' between
object-oriented programming and the relational model, we see
considerable opportunity in adapting existing database concurrency
control to better serve these communities---via both increased native
support for declarative invariants and server-side code execution.

In summary, this paper makes the following contributions:
\begin{myitemize}
\item We analyze 67 open source Ruby on Rails applications to
  determine their use of both database-backed and feral concurrency
  control mechanisms.

\item We analytically and experimentally quantify the degree of
  inconsistency allowed by Rails's built-in uniqueness and association
  validations.

\item We survey six additional popular frameworks for similarly unsafe
  validations and discuss the implications of feral concurrency
  control for future database system designs.
\end{myitemize}

The remainder of this paper proceeds as
follows. Section~\ref{sec:motivation} briefly provides background on
Rails MVC and deployment, while Section~\ref{sec:rails-cc} surveys
Rails's supported concurrency control
mechanisms. Section~\ref{sec:apps} presents analysis of mechanism
usage in open source applications as well as safety under weak
isolation.  Section~\ref{sec:evaluation} experimentally quantifies the
degree of inconsistency allowed by these mechanisms in a Rails
deployment. Section~\ref{sec:other-orms} discusses additional web
frameworks, and Section~\ref{sec:discussion} presents recommendations
for better supporting these framework demands. Section~\ref{sec:relatedwork}
presents related work and Section~\ref{sec:conclusion} concludes.
