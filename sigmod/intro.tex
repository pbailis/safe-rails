
\section{Introduction}
\label{sec:intro}

The rise of ``Web 2.0'' Internet applications delivering dynamic,
highly interactive user experiences has been accompanied by a new
generation of programming frameworks~\cite{web20}. These frameworks
simplify common tasks such as content templating and presentation,
request handling, and, notably, data storage, allowing developers to
focus on ``agile'' development of their applications. This trend
embodies the most recent realization of the larger vision of
object-relational mapping (ORM) systems~\cite{orm-db}, albeit at a
unprecedented scale of deployment and programmer adoption.

As a lens for understanding this modern ORM behavior, we study Ruby on
Rails (or, simply, ``Rails'')~\cite{rails-book,rails-computer}, a
central player among modern frameworks powering sites including (at
one point) Twitter~\cite{twitter-rails}, Airbnb~\cite{airbnb-rails},
GitHub~\cite{github-rails}, Hulu~\cite{hulu-rails},
Shopify~\cite{shopify-rails}, Groupon~\cite{groupon-rails},
SoundCloud~\cite{soundcloud-rails}, Twitch~\cite{twitch-rails},
Goodreads~\cite{goodreads-rails}, and
Zendesk~\cite{zendesk-rails}. From the perspective of database systems
research, Rails is interesting for at least two reasons. First, it
continues to be a popular means of developing responsive web
application front-end and business logic, with an active open source
community and user base. Rails recently celebrated its tenth
anniversary and enjoys considerable commercial interest, both in terms
of deployment and the availability of hosted ``cloud'' environments
such as Heroku. Thus, Rails programmers represent a large class of
consumers of database technology. Second, and perhaps more
importantly, Rails is ``opinionated
software''~\cite{dhh-opinionated}. That is, Rails embodies the strong
personal convictions of its developer community, and, in particular,
David Heinemeier Hansson (known as DHH), its creator. Rails is particularly
opinionated towards the database systems that it tasks with data
storage. To quote DHH:
\begin{quote}
``I don't \textit{want} my database to be clever! \dots I consider stored procedures and constraints vile and reckless destroyers of coherence. No, Mr. Database, you can not have my business logic. Your procedural ambitions will bear no fruit and you'll have to pry that logic from my dead, cold object-oriented hands \dots I want a single layer of cleverness: My domain model.''~\cite{dhh-clever}
\end{quote}
Thus, this wildly successful software framework bears an actively
antagonistic relationship to database management systems, echoing a familiar refrain of the ``NoSQL'' movement: get the database out of the way and let the application do the work.

In this paper, we examine the implications of this impedance mismatch
between databases and modern ORM frameworks in the context of
application integrity. By shunning decades of work on native database
concurrency control solutions, Rails has developed a set of primitives
for handling application integrity in the application tier---building,
from the underlying database system's perspective, a \textit{feral}
concurrency control system. We examine the design and use of these
feral mechanisms and evaluate their effectiveness in practice by
analyzing them and experimentally quantifying data integrity
violations in practice. Our goal is to understand how this growing
class of applications currently interacts with database systems and
how we, as a database systems community, we can positively engage with
these criticisms to better serve the needs of these developers.

We begin by surveying the state of Rails' application-tier concurrency
control primitives and examining their use in 67 open source
applications representing a variety of use cases from e-Commerce to
Customer Relationship Management and social networking. We find that,
these applications overwhelmingly use Rails' built-in support for
declarative invariants---\textit{validations} and
\textit{associations}---to protect data integrity---instead of
application-defined transactions, which are used more than 37 times less
frequently. Across the survey, we find over $9950$ uses of
application-level validations designed to ensure correctness criteria
including referential integrity, uniqueness, and adherence to common
data formats.

Given this corpus, we subsequently ask: are these feral invariants
correctly enforced? Do they work in practice? Rails will execute
validation checks concurrently, so we study the potential for data
corruption due to races if validation and update activity does not run
within a serializable transaction in the database. This is a real
concern, as many DBMS platforms use non-serializable isolation by
default and in many cases (despite labeling otherwise) do not provide
serializable isolation as an option at all.  Accordingly, we apply
invariant confluence analysis~\cite{coord-avoid} and show that, in
fact, up to $86.9\%$ of Rails validation usage by volume is actually
safe under concurrent execution. However, the remainder---which
include uniqueness violations under insertion and foreign key
constraint violations under deletion---are not. Therefore, we quantify
the impact of concurrency on data corruption for Rails uniqueness and
foreign key constraints under both worst-case analysis and via actual
Rails deployment. We demonstrate that, for pathological workloads,
validations reduce the severity of data corruption by orders of
magnitude but nevertheless still permit serious integrity violations.

Given these results, we return to our goal of improving the underlying
data management systems that power these applications and present
recommendations for the database research community. We expand our
study to survey several additional web frameworks and demonstrate that
many also provide a notion of feral validations, suggesting an
industry-wide trend. While the success of Rails and its ilk---despite
(or perhaps due to) their aversion to database technology---are firm
evidence of the continued impedance mismatch between object-oriented
programming and the relational model, we see considerable opportunity
in improving database systems to better serve these communities---via
more programmer- and ORM-friendly interfaces that ensure correctness
while minimizing impacts on performance and portability.

In summary, this paper makes the following contributions:
\begin{myitemize}
\item We analyze 67 open source Ruby on Rails applications to
  determine their use of both database-backed and feral concurrency
  control mechanisms. This provides a quantitative picture of how
  mainstream web developers interact with database systems, and, more
  specifically, concurrency control.

\item We study these applications' feral mechanisms potential for
  application integrity violations. We analytically and experimentally
  quantify the incidence and degree of inconsistency allowed by
  Rails's uniqueness and association validations.

\item We survey six additional frameworks for similarly unsafe
  validations. Based on these results and those above, we present a
  set of recommendations for database systems designers, including
  increasing database support for application invariants while avoiding
  coordination and maintaining portability.
\end{myitemize}

In all, this paper is an attempt to understand how a large and growing
class of programmers and framework authors interacts with the data
management systems that this community builds. We hope to raise
awareness about prevalent and under-supported application programming
patterns and their impact on the integrity of real-world, end-user
database-backed applications. Our contributions do not include a new
system for database concurrency control; rather, our goal is to inform
designers and architects of next-generation data management systems
and provide quantitative evidence of the practical shortcomings and
pitfalls in real-world database concurrency control today. We view
this work as an early example of the promising opportunity in
empirical analysis of open source database-backed software as written and deployed in practice.

The remainder of this paper proceeds as
follows. Section~\ref{sec:motivation} briefly provides background on
Rails MVC and deployment, while Section~\ref{sec:rails-cc} surveys
Rails's supported concurrency control
mechanisms. Section~\ref{sec:apps} presents analysis of mechanism
usage in open source applications as well as safety under weak
isolation.  Section~\ref{sec:evaluation} experimentally quantifies the
integrity violations allowed in a Rails
deployment. Section~\ref{sec:other-orms} describes support for feral
validations in additional frameworks, and Section~\ref{sec:discussion}
presents recommendations for better supporting these framework
demands. Section~\ref{sec:relatedwork} presents related work, and
Section~\ref{sec:conclusion} concludes with a reflection on the
potential of empirical methods in database systems research.
