
\section{Isolation and Integrity}
\label{sec:apps}

We now turn our attention to understanding which of Rails' feral
validations and associations are actually correct under concurrent
execution as described in Section~\ref{sec:deployment} and which
require stronger forms of isolation.

\subsection{Understanding Validation Behavior}

To begin, we discuss validation behavior in the abstract. Recall that
each of a model's declared validations is run before a model instance
is saved to the database. Correctly enforcing validations entails
ensuring concurrent validation execution preserves the intention of
the declared validation.

\minihead{Isolated validations} If each sequence of validations (and, if validations pass, model
update) is wrapped within a database-backed transaction, under
serializable isolation, validations would appear to execute
correctly. However, relational database engines often default to
non-serializable isolation~\cite{hat-vldb}; notably, PostgreSQL and
MySQL actually default to respectively, the weaker Read Committed and
Repeatable Read isolation levels.

We did not encounter evidence that applications changed the isolation
level. Rails does not configure the database isolation level for
validations, and none of the application code or configurations we
encountered change the default isolation level, either (or mention
doing so in documentation). Thus, although we cannot prove that this
is indeed the case, this data suggests that validations are likely
to run at default database isolation in production environments.

\minihead{Validations without isolation} Given that validations are
not likely to be perfectly isolated, does this lack of serializable
isolation actually affect these invariants?  Just because validations
effectively run concurrently does not mean that they are necessarily
incorrect. To determine exactly which of these invariants are correct
under concurrent execution, we draw on the recently developed theory
of invariant confluence~\cite{coord-avoid}.

Invariant confluence (I-confluence) provides a necessary and
sufficient condition for whether or not invariants can be preserved
under coordination-free, concurrent execution of
transactions. Informally, the condition ensures that, if transactions
maintain database states that are correct with respect to an invariant
when run in isolation, the transactions can be run simultaneously and
their results combined (``merged'') to produce another correct
state. In the context of a Rails application running on a MySQL or
PostgreSQL back-end, validations are already preserved in isolation by
Rails (as the model will not be saved otherwise). However, we need to
ensure that, in the event of concurrent validations and model saves,
the result of concurrent model saves will not violate the validation
for either model. In the event that two concurrent controllers save
the same model (backed by the same database record), only one will be
persisted (a some-write-wins ``merge''). In the event that two
concurrent controllers save different models (i.e., backed by
different database records), both will be persisted (a set-based
``merge''). In both cases, we must ensure that validations hold after
merge.

Per~\cite{coord-avoid}, the state of the art in I-confluence
analysis currently relies on a combination of manual proofs and simple
static analysis: given a set of invariant and operation pairs
classified as providing the I-confluence property, we can
iterate through all operations and declared invariants and check
whether or not they appear in the set of I-confluent pairs. If
so, we can label the pair as I-confluent. If not, we can
either conservatively label the pair as unsafe under concurrent
execution or prove the pair as I-confluent or not. (To prove a
pair is I-confluent, we must show that the set of database
states reachable by executing operations preserves the invariant under
merge, as described in the paragraph above.)

Returning to our task of classifying Rails validations and
associations as safe or not, we applied this I-confluence analysis to
the validations in the corpus we examined (see also
Appendix~\ref{sec:appendix-ic}). Recall that Rails provides
support for arbitrary, user-defined validation functions. In fact, in
our analysis, we found that only 60 out of 3551 validations were
expressed as user-defined functions. The remainder were drawn from the
standard set of validations supported by Rails core.\footnote{It is
  unclear exactly why this is the case. It is possible that, because
  these invariants are standardized (in our case, in Rails, and,
  in~\cite{coord-avoid}, in SQL), they are more accessible to
  users. It is also possible that SQL and Rails have simply done a
  good job of codifying common patterns that programmers tend to
  use. This chicken-and-egg problem is an interesting subject for
  further methodological study.} Accordingly, we begin by considering
built-in validations, then examine each of the custom validations.

\subsection{Built-In Validations}

We now discuss common, built-in validations and their
I-confluence. Many are I-confluent and are therefore safe to execute
concurrently.

Table~\ref{table:builtins} presents the eleven most common built-in
validations by usage and their occurences in our application
corpus. The exact coordination requirements depended on their usage.

The most popular invariant, \texttt{presence}, serves multiple purposes. Its
basic behavior is to simply check for empty values in a model before
saving. This is I-confluent as, in our model, concurrent model saves
cannot result in non-null values suddenly becoming null. However,
\texttt{presence} can also be used to enforce that the opposite end of
an association is, in fact, present in the database (i.e., referential
integrity). Under insertions, foreign key constraints are
I-confluent~\cite{coord-avoid}, but, under deletions, they are not.

The second most popular invariant, concerning record uniqueness, is
\textit{not} I-confluent~\cite{coord-avoid}. That is, if two users
concurrently insert or modify records, they can introduce duplicates.

Eight of the next nine invariants are largely concerned with data
formatting and are I-confluent. For example, \texttt{numericality}
ensures that the field contains a number rather than an alphanumeric
string. These invariants are indeed I-confluent under concurrent
update.

Finally, \texttt{associated} (like \texttt{presence}) depends on
whether or not the current updates are both insertions (I-confluent)
or mixed insertions and deletions (I-confluent).

Overall, a large number of built-in validations are safe under
concurrent operation. Under insertions, 87.3\% of built-in validation
occurrences as I-confluent. Under deletions, only 35.2\% of
occurrences are I-confluent.  However, associations and multi-record
uniqueness are---depending on the workload---not I-confluent and are
therefore likely to cause problems. In the next section, we examine
these validations in greater detail.

\begin{table}
\begin{center}
\small
\begin{tabular}{|l l l |}
\hline
Name & Occurrences & I-Confluent?\\\hline
\texttt{validates\_presence\_of} & 1764 & Depends\\
\texttt{validates\_uniqueness\_of} & 442 & No \\
\texttt{validates\_length\_of} & 302 & Yes \\
\texttt{validates\_inclusion\_of} & 167 & Yes\\
\texttt{validates\_length} & 138 & Yes \\
\texttt{validates\_format\_of} & 118 & Yes\\
\texttt{validates\_numericality\_of} & 137 & Yes \\
\texttt{validates\_format} & 69 & Yes \\
\texttt{validates\_associated} & 41 & Depends\\
\texttt{validates\_inclusion} & 36 & Yes \\
\texttt{validates\_email} & 34 & Yes \\
Other & 303 & \\\hline
\end{tabular}
\end{center}\vspace{-1em}
\caption{Use of and invariant confluence of built-in validations.}
\label{table:builtins}
\end{table}

\subsection{Custom Invariants}

We also manually inspected the coordination requirements of the 60
(1.71\%) invariants (from 17 projects) that were declared as UDFs. 52
of these custom validations were declared inline via Rails's
\texttt{validates\_each} syntax, while 8 were custom classes that
implemented Rails's validation interface. 42 of 60 validations were
I-confluent, while the remaining 18 were not. For brevity, we omit a
discussion of each validation (following double-blind review, we plan
to open-source all analysis; for the time being, it \textit{is}
possible to view these validations in the wild using the open source
projects and hashes from Table~\ref{table:app-summary}) but discuss
several trends and notable examples below.

Among the custom validations that were I-confluent, many consisted of
simple format checks or other domain-specific validations, including
credit card formatting and static username blacklisting.

The validations that were not I-confluent took on a range of
forms. Three validations performed the equivalent of foreign key
checking, which, as we have discussed, is unsafe under deletion. Three
validations checked database-backed configuration options including
the maximum allowed file upload size and default tax rate; while
configuration updates are ostensibly rare, the outcome of each
validation could be affected under a configuration change. Two
validations were especially interesting. Spree's
\texttt{AvailabilityValidator} checks whether an eCommerce inventory
has sufficient stock available to fulfill an order; concurrent order
placement might result in negative stock. Discourse's
\texttt{PostValidator} checks whether a user has been spamming the
forum; while not necessarily critical, a spammer could technically
foil this validation by attempting to simultaneously author many posts.

In summary, again, a large proportion of validations appear
safe. Nevertheless, the few non-I-confluent validations should be
cause for concern under concurrent execution.


