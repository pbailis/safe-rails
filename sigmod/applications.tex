
\section{Analytical Integrity Analysis}
\label{sec:apps}

We now turn our attention to understanding which of Rails' feral
validations and associations are actually correct under the execution
model discussed in Section~\ref{sec:deployment}. 

Recall that each of a model's declared validations is run before a model instance is saved to
the database. To correctly enforce a validation requires either that
\textit{i.)} the validations are isolated from one another or
that \textit{ii.)} the validations are somehow ``safe'' to run
concurrently.

Are validations isolated? Given that each sequence of validations is
wrapped within a transaction, under serializable isolation,
validations would appear to execute correctly. However, as is common
in relational database engines~\cite{hat-vldb}, neither PostgreSQL nor
MySQL actually default to serializable isolation, and instead provide,
respectively, the weaker Read Committed and Repeatable Read isolation
levels. Under this isolation level, as we will see shortly, many
validations effectively run concurrently. While Rails 4 does provide
support for changing the isolation level on a per-transaction basis,
Rails does not actually change the database isolation level for
validations. Similarly, none of the application code or configurations
actually changethe default isolation level. Moreover, we did not
encounter any application deployment documentation that suggested
changing the isolation level. Although we cannot prove that this is
the case, this data suggests that validations are likely running at
default isolation in practice.

Does a lack of serializable isolation actually affect these
invariants? Just because validations effectively run concurrently does
not mean that they are necessarily incorrect. To determine exactly
which of these invariants are correct under concurrent execution, we
draw on the recently developed theory of invariant
confluence~\cite{coord-avoid}.

Informally, invariant confluence provides a necessary and sufficient
condition for whether or not invariants can be preserved under
coordination-free, concurrent execution of transactions. The condition
effectively captures the property that ``the set of [invariant] valid
states reachable by executing transactions and merging their results
is closed (w.r.t. validity) under merge [of divergent states].'' In
our MySQL and PostgreSQL back-ends, ``merge'' of two concurrent Rails
validation transactions consists of either $i.)$ in the case of two
concurrent inserts or updates to model instances with different IDs,
placing the two models in the same table or, $ii.)$ in the case of two
concurrent inserts or updates to model instances with the same IDs,
choosing an arbitrary ``winning'' write. That is, if operations $o_1$
and $o_2$ attempt to concurrently create new model instances $i_1$ and
$i_2$ with different IDs, if each of $o_1$ and $o_2$'s validations
pass, we will end up with two new entries in the model table in the
database. If $i_1$ and $i_2$ have the same ID, we will end up with
only one of the models in the database. In the latter case, we end up
with a Lost Update, but, in general, we find that the invariants that
are subject to violations under update are similarly subject to
anomalies under insertion.

Per~\cite{coord-avoid}, the state of the art in invariant confluence
analysis currently relies on a combination of manual proofs and simple
static analysis. Given a set of invariant and operation pairs
classified as providing the invariant confluence property, we can
iterate through all operations and declarated invariants and check
whether or not they appear in the set of invariant confluent pairs. If
so, we can label the pair as invariant confluent. If not, we can
conservatively label the pair as unsafe under concurrent execution.

Returning to our task of classifying Rails validations and
associations, we applied this invariant confluence analysis to the
validations in the corpus we examined. As~\cite{coord-avoid} observed,
many applications re-use a common set of invariants. We observed a
similar trend in our analysis of Rails applications. Recall that Rails
provides support for arbitrary, user-defined validation functions. In
fact, in our analysis, we found that only 60 out of 3551 validations
were expressed as user-defined functions. The remainder were drawn
from the standard set of validations supported by Rails
core.\footnote{It is unclear exactly why this is the case. It is
  possible that, because these invariants are standardized (in our
  case, in Rails, and, in~\cite{coord-avoid}, in SQL), they are more
  accessible to users. It is also possible that SQL and Rails have
  simply done a good job of codifying common patterns that programmers
  tend to use. This chicken-and-egg problem is an interesting subject
  for further methodological study.} Accordingly, we begin by
considering built-in validations, then examine each of these custom
validations.

\subsection{Built-In Validations}

Table~\ref{table:builtins} presents the eleven most common built-in
validations by usage and their occurences in our application
corpus. The most popular, \texttt{presence} is multi-modal: its basic
behavior is to simply check for empty values in a model before
saving. However, it can also be used to enforce that the opposite end
of an association is, in fact, present in the database (i.e.,
referential integrity). The former use case is invariant confluent,
while the latter depends on whether or not the codebase uses deletions
or not. The second most popular invariant, \texttt{uniqueness} is
\textit{not} invariant confluent. That is, if two users concurrently
insert or modify records, they can introduce duplicates. Eight of the
next nine invariants are largely concerned with data formatting---for
example, \texttt{numericality} ensures that the field contains a
number rather than an alphanumeric string. These invariants are indeed
invariant confluent under concurrent update. Finally,
\texttt{associated} depends on whether or not the current updates are
both insertions (safe) or mixed insertions and deletions
(unsafe). Ignoring \texttt{presence} and \texttt{associated}, we can
label 74.7\% of validation occurrences as invariant
confluent. Including these two invariants, if we mark them as not
invariant confluent (i.e., consider mixed deletions and insertions),
we have 35.2\% occurrence of invariant confluence, and, optimistically
(only considering insertions), a 87.3\% occurence of invariant
confluence invariants.

Overall, a large number of built-in validations are safe under
concurrent operation. However, associations and multi-record
uniqueness are---depending on the workload---not invariant confluent
and are therefore likely to cause problems. In the next section, we
examine these invariants in greater detail.

\begin{table}
\begin{tabular}{|l l l |}
\hline
Name & Occurrences & I-Confluent?\\\hline
\texttt{validates\_presence\_of} & 1764 & Depends\\
\texttt{validates\_uniqueness\_of} & 442 & No \\
\texttt{validates\_length\_of} & 302 & Yes \\
\texttt{validates\_inclusion\_of} & 167 & Yes\\
\texttt{validates\_length} & 138 & Yes \\
\texttt{validates\_format\_of} & 118 & Yes\\
\texttt{validates\_numericality\_of} & 137 & Yes \\
\texttt{validates\_format} & 69 & Yes \\
\texttt{validates\_associated} & 41 & Depends\\
\texttt{validates\_inclusion} & 36 & Yes \\
\texttt{validates\_email} & 34 & Yes \\
Other & 303 & \\\hline
\end{tabular}
\caption{Use of and invariant confluence of built-in validations.}
\label{table:builtins}
\end{table}

\subsection{Custom Invariants}

We also manually inspected the coordination requirements of the 60
(1.71\%) invariants (from 17 projects) that were declared as UDFs. 52
of these were declared inline via Rails's \texttt{validates\_each}
syntax, while 8 were custom validator classes that implemented Rails's
validator interface. 42 of 60 validations were invariant confluent,
while the remaining 18 were not. For brevity, we omit a discussion of
each validator (following double-blind review, we plan to open-source
all analysis; for the time being, it \textit{is} possible to view
these validators in the wild using the open source projects and hashes
from Table~\ref{table:app-summary}) but discuss several trends and
notable examples below.

Among the invariant confluent custom validations, many consisted of
simple format checks or other domain-specific validation, including
credit card formatting and static username blacklisting.

The non-invariant confluent validations took on a range of
forms. Three validations performed the equivalent of foreign key
checking, which, as we have discussed, is unsafe under deletion. Three
validations checked database-backed configuration options including
the maximum allowed file upload size and default tax rate; while
configuration updates are ostensibly rare, the outcome of each
validation could be affected under a configuration change. Two
validations were especially interesting. Spree's
\texttt{AvailabilityValidator} checks whether an eCommerce inventory
has sufficient stock available to fulfill an order; concurrent order
placement might result in negative stock. Discourse's
\texttt{PostValidator} checks whether a user has been spamming the
forum; while not necessarily critical, a spammer could technically
foil this validation by attempting to simultaneously author many posts.

In summary, again, a large class of validations appear safe. Nevertheless,
these few examples highlight the importance of protecting against
these anomalies. 


