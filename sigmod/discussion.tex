
\section{Other Frameworks}
\label{sec:other-orms}

While our focus in this paper is on Rails, we briefly investigated
support for uniqueness, foreign key, and custom validations in several
other frameworks. We find widespread support for validations and
varying susceptibility to integrity errors.

\newcommand{\orm}[1]{{\vspace{.45em}\noindent\textit{#1}}}

\orm{Java Persistence API} (JPA; version EE 7)~\cite{code-jpa} is a
standard Java Object persistence interface and supports both
uniqueness and primary key constraints in the database via specialized
object annotations. Thus, when JPA is used to create a table, it will
use the database to enforce these constraints. In 2009, JPA introduced
support for UDF validations via a JavaBean
interface~\cite{code-bean-validation}. Interestingly, both the
original (and current) Bean validation specifications specifically
address the use of uniqueness validations in their notes:
\begin{quote}
``Question: should we add @Unique that would map to @Column(unique=true)?
@Unique cannot be tested at the Java level reliably but could generate
a database unique constraint generation. @Unique is not part
of the [Bean Validation] spec today.''~\cite{jsr-bean}
\end{quote}
An author of a portion of the code specification notes separately:
\begin{quote}
  ``The reason @Unique is not part of the built-in constraints is the
  fact that accessing the [database] during a valiation [sic] is
  opening yourself up for potenital [sic] phantom reads. Think twice
  before you go for [an application-level] approach.''~\cite{unique-bean}
\end{quote}
%http://download.oracle.com/otn-pub/jcp/persistence-2.0-fr-eval-oth-JSpec/persistence-2_0-final-spec.pdf?AuthParam=1414639627_4654729130380ec6809634192c3faacb
By default, JPA Validations are run upon model save and run in a
transaction at the default isolation level, and therefore, as the
developers above hint, are susceptible to the same kinds of integrity
violations we study here.

\orm{Hibernate} (version 4.3.7)~\cite{code-hibernate}, a Java ORM
based on JPA, does \textit{not} automatically enforce declared foreign
key relationships: if a foreign key constraint is declared; a
corresponding column is added, but that column is \textit{not} backed by a
database foreign key. Instead, for both uniqueness and foreign key
constraints, Hibernate relies on JPA schema annotations for
correctness. Therefore, without appropriate schema annotations,
Hibernate's basic associations may contain dangling references. Hibernate also has an extensive user-level validation
framework implementing on the JPA Validation Bean
specification~\cite{code-hibernate-validator} and is sensitive to weak
isolation anomalies, similar to Rails validations.

\orm{CakePHP} (version 2.5.5)~\cite{code-cakephp}, a PHP-based web
framework, supports uniqueness, foreign key, and UDF
validations. CakePHP does \textit{not} back any of its validation
checking with a database transaction and relies on the user to
correctly specify any corresponding foreign keys or uniqueness
constraints within the database the schema. Thus, while users can
declare each of these validations, there is no guarantee that they are
actually enforced by the database. Thus, unless users are careful to
specify constraints in both their schema and in their validations,
validations may lead to integrity violations.

\orm{Laravel} (version 4.2)~\cite{code-laravel}, another PHP-based web
framework, supports the same set of functionality as CakePHP,
including application-level uniqueness, foreign key, and UDF
validations in the application. Any database-backed constraints must
be specified manually in the schema. Per one set of community
documentation~\cite{laravel-book}, ``database-level validations can
efficiently handle some things (such as uniqueness of a column in
heavily-used tables) that can be difficult to implement otherwise''
but ``[t]esting and maintenance is more difficult...[and] your
validations would be database- and schema-specific, which makes
migrations or switching to another database backend more difficult in
the future.'' In contrast, model-level validations are ``the
recommended way to ensure that only valid data is saved into your
database. They are database agnostic, cannot be bypassed by end users,
and are convenient to test and maintain.'' 

\orm{Django} (version 1.7)~\cite{code-django}, a popular Python-based
framework, backs declared uniqueness and foreign key constraints with
database-level constraints. It also supports custom validations, but
these validations are not wrapped in a
transaction~\cite{code-django-constraints}. Thus, Django also appears
problematic, but only for custom validations.

\orm{Waterline} (version 0.10), a persistence layer for
node.js~\cite{code-waterline} (backing Sails.js, a popular MVC
framework written in Node.js~\cite{code-sails}), provides support for
in-DB foreign key and uniqueness constraints (when supported by the
database) as well as custom validations (that are \textit{not}
supported via transactions; e.g., ``TO-DO: This should all be wrapped
in a transaction. That's coming next but for the meantime just hope we
don't get in a nasty state where the operation
fails!''~\cite{code-waterline-txn}).

\minihead{Summary} In all, we observe wide, cross-framework support
for feral validation/invariants, with inconsistent use of mechanisms
for enforcing them ranging from the use of in-database constraints to
transactions to no ostensible use of concurrency control in either
application or database.

% Implementing and using unique indexes~\cite{waterline-unique-one,waterline-unique-two}.

\begin{comment}

Name & PK & FK & UDF Validations
DJANGO & Automatic & Automatic & Yes
JPA & Yes, annotation & Yes, annotation & Yes, via Beans
CakePHP & 

% Django

Supports validations
https://docs.djangoproject.com/en/dev/ref/validators/

Broken online: http://stackoverflow.com/a/5690705


1.9 (not yet released) deprecating in favor of checks
https://docs.djangoproject.com/en/1.7/internals/deprecation/

checks:
https://docs.djangoproject.com/en/1.7/topics/checks/



%FK 

does it right
https://docs.djangoproject.com/en/1.7/ref/models/fields/#django.db.models.ForeignKey

%PK

does it right, by default declares an index
https://docs.djangoproject.com/en/1.7/_modules/django/db/backends/schema/


'''
If True, this field must be unique throughout the table.

This is enforced at the database level and by model validation. If you try to save a model with a duplicate value in a unique field, a django.db.IntegrityError will be raised by the model’s save() method.

This option is valid on all field types except ManyToManyField, OneToOneField, and FileField.

Note that when unique is True, you don’t need to specify db_index, because unique implies the creation of an index.
'''
a

% Hibernate/JPA


%FK 

no FK by default, relies on JPA
http://docs.jboss.org/hibernate/orm/4.3/manual/en-US/html/ch08.html

JoinColumn
http://docs.oracle.com/javaee/6/api/javax/persistence/OneToMany.html

%PK

``The reason @Unique is not part of the built-in constraints is the fact that accessing the Session/EntityManager during a valiation is opening yourself up for potenital phantom reads. Think twice before you go for the following approach.''
https://developer.jboss.org/wiki/AccessingtheHibernateSessionwithinaConstraintValidator?_sscc=t
http://stackoverflow.com/questions/17092601/validate-unique-username-in-spring

http://docs.oracle.com/javaee/6/api/javax/persistence/UniqueConstraint.html

BEAN supported
https://jcp.org/en/jsr/detail?id=303

%CakePHP
supports udf validations
http://book.cakephp.org/2.0/en/models/data-validation.html

basically you set up your own schema, but default validations don't
appear to be transactional

%FK 


%PK

https://github.com/cakephp/cakephp/blob/50b3893e6507979427e1aaeb435494aed1af4f52/lib/Cake/Model/Model.php#L3303


% Laravel

Manually define database schema!

%FK 
%PK

http://laravel.com/docs/4.2/validation#rule-unique
https://github.com/laravel/framework/blob/75b1dff27778354e44511556171cf6ae466c8b59/src/Illuminate/Validation/Validator.php#L940


http://laravelbook.com/laravel-input-validation/

% Node -- Sail.js
%FK 
%PK

Validations are handled by Anchor, a thin layer on top of Validator, one of the most robust validation libraries for Node.js. Sails supports most of the validations available in Validator, as well as a few extras that require database integration, like unique.

http://sailsjs.org/#/documentation/concepts/ORM/Validations.html
https://github.com/balderdashy/sails/issues/832

https://github.com/balderdashy/waterline

Broken in Mongo
https://github.com/balderdashy/sails-mongo/issues/152

Broken in dev
https://github.com/balderdashy/waterline/issues/55

Because you have migrate: safe set the indexes will not be created when you start the ORM.
https://github.com/balderdashy/waterline/issues/236

uses db foreign keys

\end{comment}

% what can we learn?
% udfs


\section{Implications for Databases}
\label{sec:discussion}

In light of this empirical evidence of the continued mismatch between
ORM applications and databases, in this section, we reflect on the
core database limitations for application writers today and suggest a
set of directions for alleviating them.

\subsection{Summary: Shortcomings Today}

The use of feral invariants is not well-supported by today's
databases. At a high level, today's databases effectively offer two
primary options for ORM framework developers and users:

\begin{impenumerate}
\item \textbf{Use ACID transactions.} Serializable transactions would
  be sufficient to correctly enforce arbitrary database
  invariants. This is core to the transaction concept: isolation is a
  means towards preserving integrity. \vspace{.5em}

  Unfortunately, in practice, for application developers, transactions
  are problematic. Given serializability's performance and
  availability overheads~\cite{brewer-cap}, developers at scale have
  largely eschewed the use of serializable transactions (which are
  anyway not required for correct enforcement of approximately 75\% of
  the invariants we encountered in the Rails corpus). Moreover, many
  databases offering ``ACID'' semantics do not provide serializability
  by default and often, even among industry-standard enterprise
  offerings, do not offer it as an option at all~\cite{hat-vldb} (to
  say nothing of implementation difficulties, as in
  Footnote~\ref{fn:pg-bug}). Instead, developers using these systems
  today must manually reason about a host of highly technical, often
  obscure, and poorly understood weak isolation models expressed in
  terms of low-level read/write anomalies such as Write Skew and Lost
  Update~\cite{adya-isolation,consistency-borders}. We have observed
  (e.g., Footnote~\ref{fn:si-rails}) that ORM and expert application
  developers are familiar with the prevalence of weak isolation, which
  may also help explain the relative unpopularity of transactions
  within the web programming community.

\item\textbf{Custom, feral enforcement.} Hand-rolling
  user-level concurrency control solutions on a per-framework or,
  worse, per-application basis is an expensive, error-prone, and
  difficult process that neglects decades of contributions from the
  database s community. While this solution is sufficient to maintain
  correctness in the 75\% (I-confluent) invariants in our corpus, the
  remainder can---in modern ORM implementations---lead to data
  corruption on behalf of applications. \vspace{.5em}

  However, and perhaps most importantly, this feral approach preserves
  a key tenet of the Rails philosophy: a recurring insistence on
  expressing domain logic in the application. This also enables the
  declaration of invariants that are not among the few natively
  supported by databases today (e.g., uniqueness constraints).
\end{impenumerate}

In summary, application writers today lack a solution that guarantees
correctness while maintaining high performance \textit{and}
programmability. Serializability is too expensive for some
applications, is not widely supported, and is not necessary for many
application invariants. Feral concurrency control is often less
expensive, is trivially portable, but is not sufficient for many other
application invariants. In neither case does the database respect and
assist with application programmer desires for a clean, idiomatic
means of expressing correctness criteria in domain logic. We believe
there is an opportunity and pressing need to build systems that
provide all three criteria: performance, correctness, and
programmability.

\subsection{Domesticating Feral Mechanisms}

Constructively, to properly provide database support and thereby
``domesticate'' these feral mechanisms, we believe application users
and framework authors need a new interface to the database that will
enable them to:
\begin{interfaceenumerate} 
\item \textit{Express correctness criteria in the language of their
    domain model, with minimal friction, while permitting their
    automatic enforcement.} Per Section~\ref{sec:motivation}, a core
  factor behind the success of ORMs like Rails appears to be their
  promulgation of an idiomatic programming style that ``seems right''
  for web programming. Rails' disregard for advanced database
  functionality is evidence of a continued impedance mismatch between
  applications and databases.  \vspace{.5em}

  We believe any solution to domestication must respect ORM
  application patterns and programming style, including the ability to
  specify invariants in each framework's native language. Ideally,
  database systems could enforce applications' existing feral
  invariants without modification. This is already feasible for a
  subset of invariants---like uniqueness and foreign key
  constraints---but not all. An ideal solution to domestication would
  provide universal support.

\item \textit{Only pay the price of coordination when necessary.} Per
  Section~\ref{sec:apps}, many invariants can be safely executed
  without coordination, while others cannot. The many that do not need
  coordination should not be unnecessarily penalized. \vspace{.5em}

  An ideal solution to domestication would enable applications to
  avoid coordination whenever possible would improve performance and
  operation availability. The database should facilitate this
  avoidance, thus evading common issues with serializable
  transactions.

\item \textit{Easily deploy to multiple database backends.}  ORM
  frameworks today are deployed across a range of database
  implementations, and, when deciding which database features to
  exercise, framework authors often choose the least common
  denominator for compatibility purposes.  \vspace{.5em}

  An ideal solution to domestication would preserve this
  compatibility, possibly providing a ``bolt on'' compatibility layer
  between ORM systems and databases lacking advanced functionality
  (effectively, a ``blessed'' set of mechanisms beneath the
  application/ORM that correctly enforce feral mechanisms).

\end{interfaceenumerate}
Fulfilling these design requirements would enable high performance,
correct execution, and programmability. However, doing so represents a
considerable challenge.

\minihead{Promise in the literature} The actual vehicle for
implementing this interface is an open question, but the literature
lends several clues. On the one hand, we do not believe the answer
lies in exposing additional read/write isolation or consistency
guarantees like Read Committed; these fail our requirement for an
abstraction operating the level of domain logic and, as we have noted,
are challenging for developers (and researchers) to reason about. On
the other hand, more recent proposals for invariant-based concurrency
control~\cite{redblue-new,coord-avoid} and a litany of work from prior
decades on rule-based~\cite{activedb-book} and, broadly,
semantics-based concurrency control~\cite{tamer-book} appear
immediately applicable and worth (re-)considering. Recent advances in
program analysis for extracting invariants~\cite{writes-forest} and
subroutines from imperative code~\cite{statusquo} may allow us to
programatically suggest new invariants, perform correspondence
checking for existing applications, and apply a range of automated
optimizations to legacy code~\cite{pyxis,waves}. Finally, clean-slate
language design and program analysis obviate the need for explicit
invariant declaration (thus alleviating concerns of specification
completeness)~\cite{calm,blazes,hilda}; while adoption within the ORM
community is a challenge, we view this exploration as worthwhile.

\minihead{Summary} In all, the wide gap between research and current practice is both a
pressing concern and an exciting opportunity to revisit many decades
of research on alternatives to serializability with an eye towards
current operating conditions, application demands, and programmer
practices. Our proposal here is demanding, but so are the framework
and application writers our databases serve. Given the correct
primitives, database systems may yet have a role to play in ensuring
application integrity.
