
\begin{abstract}
The rise of Web 2.0 internet services over the last ten years has led to a range of popular new programming frameworks. In this work, we examine one such framework---Ruby on Rails---and its use and abuse of database concurrency control mechanisms. Namely, Rails spurns the use of traditional transactions in lieu of application-level mechanisms including declarative validations and associations. We quantitatively analyze the use of these mechanisms in a range of open source projects and determine which actually provide integrity---and which of these application-level mechanisms may lead to data corruption.
\end{abstract}