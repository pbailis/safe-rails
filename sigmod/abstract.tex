
\begin{abstract} The rise of data-intensive Web 2.0 Internet services has led to a range of popular new programming frameworks. In this work, we examine one such framework---Ruby on Rails---and its use and abuse of database concurrency control mechanisms. Namely, Rails programmers eschew the use of traditional transaction-oriented programming in lieu of \textit{feral}, or application-level, mechanisms including declarative validations and associations. In this work, we examine the implications of this decision for both applications and database systems. We quantitatively analyze the use of these mechanisms in a range of open source projects and determine which actually ensure integrity and which of these feral mechanisms may lead to data corruption, which we experimentally quantify. We subsequently present recommendations for database system designers for reducing and eliminating these integrity violations. \end{abstract}
