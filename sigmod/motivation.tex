
\section{Background}
\label{sec:motivation}

In 2004, David Heinemieir Hansson (DHH) open-sourced Ruby on Rails (``Rails''), a web application framework that would come to power much of ``Web 2.0,'' including sites such as. Rails, which bundles an object-relational mapping (ORM, or, in Rails terms, the \textit{Model}), a presentation layer (the \textit{View}), and business logic (the \textit{Controller}), is focused on web developer productivity. Its design maxims of ``Don't Repeat Yourself'' and ``Convention over Configuration'' proved a popular choice among developers, who took advantage of the framework's  ``agility'': many common tasks, such as adding a new application Model (and performing a schema migration), are easily achieved via a few standard Rails commands.
